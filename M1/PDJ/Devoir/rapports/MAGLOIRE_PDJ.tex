\documentclass[a4paper,12pt]{article}

\usepackage{multicol}
\usepackage[francais]{babel}
\usepackage[usenames]{color}
\usepackage[latin1]{inputenc}
% maths
\usepackage{amsmath}
\usepackage{amsfonts}

\usepackage{calrsfs}
\usepackage{comment}
\usepackage{graphicx}
\usepackage[T1]{fontenc}
% pour les algos 
\usepackage{vmargin}
\usepackage{tikz}

% footer & header Usage : \[l|c|r]head{} \[l|c|r]foot{} chose one letter
\usepackage{fancyhdr}
\pagestyle{fancy}

% landscape mod usage : \begin{landscape} something \end{landscape}
\usepackage{pdflscape}

%images
\usepackage{float}
\usepackage{caption}
\usepackage{subcaption}
\restylefloat{figure}

%BibTex
\usepackage{cite}
\usepackage[nottoc, notlof, notlot]{tocbibind}

\begin{document}
% -------------------- Input du title --------------------
	
\begin{titlepage}

\newcommand{\HRule}{\rule{\linewidth}{0.5mm}} % Defines a new command for the horizontal lines, change thickness here

\center % Center everything on the page

%----------------------------------------------------------------------------------------
%	LOGO SECTION
%----------------------------------------------------------------------------------------
\begin{minipage}{0.45\textwidth}

\begin{flushleft}
\includegraphics[width=6cm]{img/Logo-IG.jpg}
\end{flushleft}
\end{minipage}
~
\begin{minipage}{0.45\textwidth}
\begin{flushright}
\includegraphics[width=6cm]{img/logop13.png} % Include a department/university logo - this will require the graphicx package
\end{flushright}
\end{minipage}\\[1.5cm]

%----------------------------------------------------------------------------------------
%	HEADING SECTIONS
%----------------------------------------------------------------------------------------

\textsc{\Huge Institut Galil�e}\\[1.25cm] % Name of your university/college
\textsc{\LARGE Programmation Distribu�e Java}\\[1cm] % Major heading such as course name
\textsc{\Large Documentation installation et utilisation}\\[0.75cm] % Minor heading such as course title
\textsc{\large Master 1 Informatique}\\[0.5cm] 

%----------------------------------------------------------------------------------------
%	TITLE SECTION
%----------------------------------------------------------------------------------------

\HRule \\[0.5cm]
{ \huge \bfseries Projet Serveur de Pizzas Java}\\[0.2cm] % Title of your document
\HRule \\[1.5cm]
 
%----------------------------------------------------------------------------------------
%	AUTHOR SECTION
%----------------------------------------------------------------------------------------

\begin{minipage}{0.45\textwidth}
\begin{flushleft} \large
\emph{Nom:}\\
Jones \textsc{Magloire} (11000369) % Your name
\end{flushleft}
\end{minipage}
~
\begin{minipage}{0.45\textwidth}
\begin{flushright} \large
\emph{Enseignant:} \\
Christophe \textsc{Fouquer�}% Supervisor's Name
\end{flushright}
\end{minipage}\\[2cm]

% If you don't want a supervisor, uncomment the two lines below and remove the section above
%\Large \emph{Author:}\\
%John \textsc{Smith}\\[3cm] % Your name

%----------------------------------------------------------------------------------------
%	DATE SECTION
%----------------------------------------------------------------------------------------

{\large \today}\\[1.5cm] % Date, change the \today to a set date if you want to be precise

%----------------------------------------------------------------------------------------

\vfill % Fill the rest of the page with whitespace

\end{titlepage}
% --------------------------------------------------------
\chead{Jones \textsc{Magloire}, Master 1 Informatique}
\rhead[]{}
\lhead[]{}
\cfoot{\thepage}
\tableofcontents
\newpage
\section{Installation de l'application} 
\paragraph{Make}
Pour l'installation, entrez dans le fichier src, lancez la commande\\
\indent \$ make

\paragraph{Clean}
Pour supprimer les fichiers de classe, lancez la commande \\
\indent \$ make clean

\paragraph{} Les dossiers qui sont fournis doivent �tre dans le dossier o� sont les classes pour que le programme puisse y acceder.

\section{Administration de l'application}
\subsection{Lancement et options}
\paragraph{Les serveurs HTTP et PIZ}
Pour le lancement utilisez la commande : \\
\indent \$ java fr.jonesalexis.project.pdj.Main [Options]\\
ou sans options avec \\
\indent \$ make start

\paragraph{Les options}
\begin{itemize}
	\item -ph [port]\\
Num�ro de port pour le serveur de requ�tes http; par d�faut : 1900
	\item -pp [port]\\
Num�ro de port pour le serveur de requ�tes piz; par d�faut : 2000
	\item -w [dossier web] \\
	Dossier web; par defaut : www 
	\item -dp [fichier BdD pizzas] \\
	Fichier Base de Donn�es des pizzas; par d�faut : db/pizzas.xml 
	\item -dt [fichier BdD types] \\
	Fichier Base de Donn�es des types; par d�faut : db/types.xml 
\end{itemize}

\subsection{Les arguments}
\paragraph{Dossier web}
Dans le dossier web, il y a toutes les pages HTML du site de pizza. Il doit n�anmoins obligatoirement contenir les fichiers \textbf{pizzas.html} et \textbf{pizzas\_end.html} o� \textbf{pizzas.html} est le d�but du fichier html � envoyer et  \textbf{pizzas\_end.html} la fin. Entre ces deux parties, du code concernant les pizzas va �tre ins�r� sous la forme: 
\begin{verbatim}
<h1>header1</h1>< /br>
<h1>header2</h1>< /br>
<p>paragraphe</p>< /br>
\end{verbatim}
Vous pouvez donc modifier votre CSS en fonction de cela. Des fichiers HTMLs sont d�j� fournis (HTML5 CSS3). 

\paragraph{Fichier de BdD pizzas} 
La liste des pizzas doit �tre sous la forme : 
\begin{verbatim}
<lesPizzas>
    <pizza type="nom du type de la pizza">
        <nom>nom de la pizza</nom>
        <description>description de la pizza</description>
    </pizza>
    ...
</lesPizzas>
\end{verbatim}

Dans la balise pizza, type correspond � l'identifiant d'un type dans le fichier des types.
On met le nom de la Pizza entre les balises de nom et sa description entre les balises description. \\
Il faut retenir que le \textbf{lien} d'une pizza qui est son nom en gardant que les lettres et les chiffres. \\
Plusieurs pizzas peuvent avoir le m�me type, mais elles ne doivent pas avoir le m�me lien. Elles doivent toutes avoir un type existant dans la liste des types.\\

\paragraph{Fichier de BdD types}
La liste des types doit �tre sous la forme :
\begin{verbatim}
<lestypes>
    <type id="nom du type">
        <prix>prix pour ce type</prix>
    </type>
    ...
</lestypes>
\end{verbatim}

Dans la balise type, id correspond � l'identifiant du type.
On met le prix de la pizza entre les balises prix.
Tous les types doivent avoir un identifiant diff�rent.

\paragraph{port} Les ports doivent �tre des entiers.

\subsection{Protocole http}
\paragraph{Fonctionnement} Le serveur HTTP va r�cup�rer les requ�tes HTTP de l'utilisateur et va lui renvoyer les pages html de la requ�te. Si une page n'existe pas, on envoie la page 404.html. Il y a uniquement les requ�tes de type GET qui sont prises en charge.

\subsection{Protocole piz}
\paragraph{Fonctionnement} Le serveur PIZ va r�cup�rer les requ�tes PIZ de l'utilisateur et va lui renvoyer les informations qu'il demande. Le serveur est constitu� d'un ma�tre qui accepte les connexions et de plusieurs esclaves qui r�pondent. Les requ�tes possibles sont : 
\begin{itemize}
\item  piz://pizza?list\textbackslash r\textbackslash n : liste tous les liens des pizzas du serveur.
\item  piz://prix?list\textbackslash r\textbackslash n : liste tous les prix de pizza du serveur.
\item  piz://type?list\textbackslash r\textbackslash n : liste tous les types de pizza du serveur.
\item  piz://[lien de pizza]?desc\textbackslash r\textbackslash n : donne la description de la pizza demand�.
\item  piz://[lien de pizza]?prix\textbackslash r\textbackslash n : donne le prix de la pizza demand�.
\end{itemize}
Il ne faut pas oublier les \textbackslash r\textbackslash n a la fin de la requ�te.

\section{Utilisation de l'application}
\subsection{Utilisation HTTP}
\paragraph{} Ouvrez votre navigateur internet, tapez http://[IP serveur]:[port] par exemple\\
 http://127.0.0.1:1900\\


\subsection{Utilisation PIZ}
\paragraph{} Pour avoir un aper�u du protocole, lancez le programme \\
\indent \$ java fr.jonesalexis.project.pdj.ClientExample \\
ou\\
\indent \$ make client\\
Il montre les requ�tes possibles et leur r�sultat.

\end{document}